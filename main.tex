\documentclass[11pt,a4paper]{report}
\usepackage[a4paper, left=20mm, right=20mm, top=30mm, bottom=20mm]{geometry}
\usepackage{amsmath,amsfonts,amssymb,amsthm,epsfig,epstopdf,titling,url,array}
\usepackage[T2A,T1]{fontenc}
\usepackage[utf8]{inputenc}
\usepackage[russian]{babel}
\usepackage{xparse}
\usepackage[shortlabels]{enumitem}

\usepackage{perpage} %the perpage package
\MakePerPage{footnote}

\usepackage{hyperref}
\hypersetup{
    colorlinks=true,
    linkcolor=blue,
    filecolor=magenta,      
    urlcolor=cyan,
}

\def\E{\mathbb{E}}
\def\Var{\mathrm{Var}}
\def\Cov{\mathrm{cov}}
\def\Corr{\mathrm{corr}}
\def\salg{\mathcal{F}}
\def\prob{\mathcal{P}}
\def\borel{\mathcal{B}}
\def\cantor{\mathcal{C}}

\def\eps{\varepsilon}
\def\Real{\mathbb{R}}
\def\Proj{\mathbb{P}}
\def\Hyper{\mathbb{H}}
\def\Integer{\mathbb{Z}}
\def\Natural{\mathbb{N}}
\def\Complex{\mathbb{C}}
\def\Rational{\mathbb{Q}}

\usepackage{stackengine,graphicx,amssymb}
\stackMath
\newcommand\frightarrow{\scalebox{1}[.3]{$\rule[.45ex]{2ex}{1.5pt}%
		\kern-.2ex{\blacktriangleright}$}}
\newcommand\darrow[1][]{\mathrel{\stackon[1pt]{\stackanchor[1pt]{\frightarrow}{\frightarrow}}{\scriptstyle#1}}}

\newcommand\independent{\protect\mathpalette{\protect\independenT}{\perp}}
\def\independenT#1#2{\mathrel{\rlap{$#1#2$}\mkern2mu{#1#2}}}
\renewcommand{\thesection}{\arabic{section}}

\theoremstyle{definition}
\newtheorem{task}{Задача}[section]

\theoremstyle{definition}
\newtheorem{theorem}{Теорема}[section]
\newtheorem{lemma}{Лемма}[section]
\newtheorem{preposition}{Утверждение}[section]
\newtheorem{corollary}{Следствие}[section]
\newtheorem{remark}{Замечание}[section]

\theoremstyle{definition}
\newtheorem{definition}{Определение}[section]
\newtheorem{example}{Пример}[section]

\makeatletter
\renewcommand\footnoterule{%
	\kern-3\p@
	\hrule\@width \textwidth
	\kern2.6\p@}
\makeatother

\title{\textbf{Теория групп, ФПМИ МФТИ}}
\author{Госткин Евгений Михайлович}
\date{}

\begin{document}
  \setlength{\parindent}{1cm}
  \maketitle
  \tableofcontents
  \newpage
  \section{Понятие группы. Примеры. Циклические группы и их подгруппы.}
  \begin{definition}\label{group_def}
  {\it Группа} - множество $G$ с операцией $*$ (умножения), обладающей следующими свойствами:
 		\begin{enumerate}[1)]
	 		\item{$\forall a, b, c \in G: (ab)c = a(bc)$ (ассоциативность);}
 			\item{$\exists e \in G~ \forall a \in G~ ae=ea=a$ (существование единицы);}
 			\item{$\forall a \in G ~\exists a^{-1}\in G:aa^{-1}=a^{-1}a=e$ (существование обратного элемента).}
 		\end{enumerate}
  \end{definition}
  \begin{definition}\label{abelian_group_def}
  	{\it Абелева группа} (коммутативная) - $\forall a, b \in G~ab=ba$.
  \end{definition}
  
	\begin{definition}\label{transformation_group_def}
	{\itГруппа преобразований} множества $X$ - совокупность $G$ его биективных преобразований, удовлетворяющая следующим условиям:
		\begin{enumerate}
			\item{$\phi, \psi\in G\Rightarrow\phi\circ\psi\in G$;}
			\item{$\phi\in G\Rightarrow\phi^{-1}\in G$;}
			\item{$id \in G$ (тождественное).}
		\end{enumerate}
	\end{definition}
 	\example{$GL_n(K):=$ группа преобразований невырожденных квадратных матриц над полем $K$ размерности $n$.}
\end{document}  
