\documentclass[11pt,a4paper]{report}
\usepackage[a4paper, left=20mm, right=20mm, top=30mm, bottom=20mm]{geometry}
\usepackage{amsmath,amsfonts,amssymb,amsthm,epsfig,epstopdf,titling,url,array}
\usepackage[T2A,T1]{fontenc}
\usepackage[utf8]{inputenc}
\usepackage[russian]{babel}
\usepackage{xparse}
\usepackage[shortlabels]{enumitem}

\usepackage{perpage} %the perpage package
\MakePerPage{footnote}

\usepackage{hyperref}
\hypersetup{
    colorlinks=true,
    linkcolor=blue,
    filecolor=magenta,      
    urlcolor=cyan,
}

\def\E{\mathbb{E}}
\def\Var{\mathrm{Var}}
\def\Cov{\mathrm{cov}}
\def\Corr{\mathrm{corr}}
\def\salg{\mathcal{F}}
\def\prob{\mathcal{P}}
\def\borel{\mathcal{B}}
\def\cantor{\mathcal{C}}

\def\eps{\varepsilon}
\def\Real{\mathbb{R}}
\def\Proj{\mathbb{P}}
\def\Hyper{\mathbb{H}}
\def\Integer{\mathbb{Z}}
\def\Natural{\mathbb{N}}
\def\Complex{\mathbb{C}}
\def\Rational{\mathbb{Q}}

\DeclareMathOperator{\ord}{ord}

\usepackage{stackengine,graphicx,amssymb}
\stackMath
\newcommand\frightarrow{\scalebox{1}[.3]{$\rule[.45ex]{2ex}{1.5pt}%
		\kern-.2ex{\blacktriangleright}$}}
\newcommand\darrow[1][]{\mathrel{\stackon[1pt]{\stackanchor[1pt]{\frightarrow}{\frightarrow}}{\scriptstyle#1}}}

\newcommand\independent{\protect\mathpalette{\protect\independenT}{\perp}}
\def\independenT#1#2{\mathrel{\rlap{$#1#2$}\mkern2mu{#1#2}}}
\renewcommand{\thesection}{\arabic{section}}

\theoremstyle{definition}
\newtheorem{task}{Задача}[section]

\theoremstyle{definition}
\newtheorem{theorem}{Теорема}[section]
\newtheorem{lemma}{Лемма}[section]
\newtheorem{preposition}{Утверждение}[section]
\newtheorem{corollary}{Следствие}[section]
\newtheorem{remark}{Замечание}[section]

\theoremstyle{definition}
\newtheorem{definition}{Определение}[section]
\newtheorem{example}{Пример}[section]

\makeatletter
\renewcommand\footnoterule{%
	\kern-3\p@
	\hrule\@width \textwidth
	\kern2.6\p@}
\makeatother

\title{\textbf{Теория групп, ФПМИ МФТИ}}
\author{Госткин Евгений Михайлович}
\date{}

\begin{document}
  \setlength{\parindent}{1cm}
  \maketitle
  \tableofcontents
  \newpage
  \section{Понятие группы. Примеры. Циклические группы и их подгруппы.}
  \begin{definition}\label{group_def}
  {\it Группа} - множество $G$ с операцией $\cdot$ (умножения), обладающей следующими свойствами:
 		\begin{enumerate}
	 		\item{$\forall a, b, c \in G: (ab)c = a(bc)$ (ассоциативность);}
 			\item{$\exists e \in G~ \forall a \in G~ ae=ea=a$ (существование единицы);}
 			\item{$\forall a \in G ~\exists a^{-1}\in G:aa^{-1}=a^{-1}a=e$ (существование обратного элемента).}
 		\end{enumerate}
  \end{definition}
  \begin{definition}\label{abelian_group_def}
  	{\it Абелева группа} (коммутативная) - $\forall a, b \in G~ab=ba$.
  \end{definition}
  
  \begin{definition}\label{subgroup_def}{\it Подгруппа} группы $G$ - $H\subset G$:
  \begin{enumerate}
  \item{$\forall a, b\in H~ab\in H$;}
  \item{$\forall a \in H ~a^{-1}\in H$;}
  \item{$e \in H$.}
  \end{enumerate}
  \end{definition}
	\begin{definition}\label{transformation_group_def}
	{\itГруппа преобразований} множества $X$ - совокупность $G$ его биективных преобразований, удовлетворяющая следующим условиям:
		\begin{enumerate}
			\item{$\phi, \psi\in G\Rightarrow\phi\circ\psi\in G$;}
			\item{$\phi\in G\Rightarrow\phi^{-1}\in G$;}
			\item{$id \in G$ (тождественное).}
		\end{enumerate}
	\end{definition}
	\begin{definition}\label{order_def}
	Для любой группы $G$ можно определить {\itстепень} элемента $g\in G$ с целым показателем:
	{\center
	\begin{equation*}
	g^k=
	\begin{cases}
		\underbrace{gg\ldots g}_{k},&k>0\\
		e, &k=0\\
		\underbrace{g^{-1}g^{-1}\ldots g^{-1}}_{k}, &k<0.
	\end{cases}
	\end{equation*}}
	\end{definition}
	\begin{preposition}\label{order_prep}$\forall g \in G~ \forall k, l \in \Integer~ g^kg^l=g^{k+l}$
	\begin{proof}Рассмотрим различные случаи для $k, l$
	\begin{enumerate}
	\item{$k, l > 0$ - очевидно}
	\item{$k > 0, l < 0, k+l>0$:
	\[
	g^kg^l=\underbrace{gg\ldots g}_{k}\underbrace{g^{-1}g^{-1}\ldots g^{-1}}_{l}=\underbrace{gg\ldots g}_{k+l}=g^{k+l}.\]}
	\end{enumerate}
	Остальные случаи рассматриваются аналогично.
	\end{proof}
	\end{preposition}
	\corollary{\label{order_prep_cor}$(g^k)^{-1} = g^{-k}$.}
	\definition{$\langle g\rangle$ - {\it циклическая подгруппа, порожденная элементом $g$} - подгруппа степеней элемента $g\in G$ (является подгруппой из определения~\ref{order_def}, утверждения~\ref{order_prep} и следствия~\ref{order_prep_cor})}
	\definition{Минимальное $m\in \Natural: g^m = e$ - {\itпорядок} элемента $g$, обозначается $\ord{g}$}

	
	\begin{example}
		$(\mathbb{Z}, +)$ - абелева группа по сложению
		\begin{itemize}
		\item{$0 \in \mathbb{Z}$ - нейтральный элемент, т.к. $\forall a \in \mathbb{Z} a + 0 = 0 + a = a$}
		\item{$\forall a \in \mathbb{Z} ~\exists a^{-1} = -a : a + (-a) = (-a) + a = 0$}
		\end{itemize}
	\end{example}
	\begin{example}
		$(\mathbb{Q}^{\times}, \cdot)$ - абелева группа по умножению, где $\mathbb{Q^\times} = \mathbb{Q}\setminus\{0\}$
		\begin{itemize}
		\item{$1 \in \mathbb{Q^\times}$ - нейтральный элемент, т.к. $\forall a \in \mathbb{Q^\times} ~a \cdot 1 = 1 \cdot a = a$}
		\item{$\forall a \in \mathbb{Q^\times} ~\exists a^{-1} = \frac{1}{a} : a\cdot\frac{1}{a} = \frac{1}{a}\cdot a = 1$}
		\end{itemize}
	\end{example}
 	\example{$GL_n(\mathbb{R})$\footnote{Название произошло от 'General linear group'.} - группа невырожденных\footnote{Для тех, кто не помнит: матрицы с ненулевым определителем.} матриц по умножению\footnote{Из курса алгема: $\forall A: \det{A} \neq 0 \Rightarrow \exists A^{-1}: A A^{-1}=A^{-1}A=E$, где $E$ - единичная, и $\det{AB} = \det{A}\cdot \det{B}$.}.}
 	\example{$SL_n[\Real] \subset GL_n[\Real]:=\footnote{Название от 'Special linear group'.} \forall A \in SL_n[\Real]~\det{A} = 1$}
 	\example{$(S_n, \circ)$\footnote{Название от 'Symmetric group'.} - группа перестановок элементов вида $\left\{1,\ldots, n\right\}$, рассматриваемых как функции $\{1,\ldots,n\}\rightarrow S_n$. $\circ$ - операция композиции функций. Является группой, т.к. есть тождественная перестановка и у каждой перестановки есть обратная. Также следует заметить, что $S_n$ подходит под определение~\ref{transformation_group_def}, поэтому можно задать действие $S_n$ на любом конечном множестве.}
 	\example{$D_{2n}$ - группа Диэдра - группа симметрий правильного $n$-угольника $A_1, \ldots, A_n$, включающая поворот и отражение. Состоит из $2n$ элементов:
 	\[
 	\{1, r, r^2, \ldots, r^{n-1}, s, sr, sr^2, \ldots, sr^{n-1}\},
 	\]
 	где $r$ - поворот $n$-угольника на $\frac{2\pi}{n}$, а $s$ - отражение относительно $OA_1$, где $O$ - центр фигуры. Таким образом, $rs$ - повернуть и отразить (читаем слева направо, как композиция функций). В частности, $r^n=s^2=1$ и $r^ks=sr^{-k}$.}
 	\example{$\{1\}$ - тривиальная группа.}
\end{document}  
