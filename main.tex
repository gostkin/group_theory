\documentclass[11pt,a4paper]{report}
\usepackage[a4paper, left=20mm, right=20mm, top=30mm, bottom=20mm]{geometry}
\usepackage{amsmath,amsfonts,amssymb,amsthm,epsfig,epstopdf,titling,url,array}
\usepackage[T2A,T1]{fontenc}
\usepackage[utf8]{inputenc}
\usepackage[russian]{babel}
\usepackage{xparse}
\usepackage[shortlabels]{enumitem}

\usepackage{perpage} %the perpage package
\MakePerPage{footnote}

\usepackage{hyperref}
\hypersetup{
    colorlinks=true,
    linkcolor=blue,
    filecolor=magenta,      
    urlcolor=cyan,
}

\def\E{\mathbb{E}}
\def\Var{\mathrm{Var}}
\def\Cov{\mathrm{cov}}
\def\Corr{\mathrm{corr}}
\def\salg{\mathcal{F}}
\def\prob{\mathcal{P}}
\def\borel{\mathcal{B}}
\def\cantor{\mathcal{C}}

\def\eps{\varepsilon}
\def\Real{\mathbb{R}}
\def\Proj{\mathbb{P}}
\def\Hyper{\mathbb{H}}
\def\Integer{\mathbb{Z}}
\def\Natural{\mathbb{N}}
\def\Complex{\mathbb{C}}
\def\Rational{\mathbb{Q}}

\DeclareMathOperator{\ord}{ord}

\renewcommand{\geq}{\geqslant}
\renewcommand{\leq}{\leqslant}
\renewcommand{\implies}{\Rightarrow}
\newcommand{\is}{\Leftrightarrow}

\usepackage{stackengine,graphicx,amssymb}
\stackMath
\newcommand\frightarrow{\scalebox{1}[.3]{$\rule[.45ex]{2ex}{1.5pt}%
		\kern-.2ex{\blacktriangleright}$}}
\newcommand\darrow[1][]{\mathrel{\stackon[1pt]{\stackanchor[1pt]{\frightarrow}{\frightarrow}}{\scriptstyle#1}}}

\newcommand\independent{\protect\mathpalette{\protect\independenT}{\perp}}
\def\independenT#1#2{\mathrel{\rlap{$#1#2$}\mkern2mu{#1#2}}}
\renewcommand{\thesection}{\arabic{section}}

\theoremstyle{definition}
\newtheorem{task}{Задача}[section]

\theoremstyle{definition}
\newtheorem{theorem}{Теорема}[section]
\newtheorem{lemma}{Лемма}[section]
\newtheorem{preposition}{Утверждение}[section]
\newtheorem{corollary}{Следствие}[section]
\newtheorem{remark}{Замечание}[section]

\theoremstyle{definition}
\newtheorem{definition}{Определение}[section]
\newtheorem{example}{Пример}[section]

\makeatletter
\renewcommand\footnoterule{%
	\kern-3\p@
	\hrule\@width \textwidth
	\kern2.6\p@}
\makeatother

\title{\textbf{Теория групп, ФПМИ МФТИ}}
\author{Госткин Евгений Михайлович}
\date{}

\begin{document}
  \setlength{\parindent}{1cm}
  \maketitle
  \tableofcontents
  \newpage
  \section{Понятие группы. Примеры. Циклические группы и их подгруппы.}
  \begin{definition}\label{group_def}
  {\it Группа} - множество $G$ с операцией $\cdot$ (умножения), обладающей следующими свойствами:
 		\begin{enumerate}
	 		\item{$\forall a, b, c \in G: (ab)c = a(bc)$ (ассоциативность);}
 			\item{$\exists e \in G~ \forall a \in G~ ae=ea=a$ (существование единицы);}
 			\item{$\forall a \in G ~\exists a^{-1}\in G:aa^{-1}=a^{-1}a=e$ (существование обратного элемента).}
 		\end{enumerate}
  \end{definition}
  \begin{definition}\label{abelian_group_def}
  	{\it Абелева группа} (коммутативная) - $\forall a, b \in G~ab=ba$.
  \end{definition}
  
  \begin{definition}\label{subgroup_def}{\it Подгруппа} $H\subset G$:
  \begin{enumerate}
  \item{$\forall a, b\in H~ab\in H$;}
  \item{$\forall a \in H ~a^{-1}\in H$;}
  \item{$e \in H$.}
  \end{enumerate}
  \end{definition}
	\begin{definition}\label{transformation_group_def}
	{\itГруппа преобразований} множества $X$ - совокупность $G$ его биективных преобразований, удовлетворяющая следующим условиям:
		\begin{enumerate}
			\item{$\phi, \psi\in G\Rightarrow\phi\circ\psi\in G$;}
			\item{$\phi\in G\Rightarrow\phi^{-1}\in G$;}
			\item{$id \in G$ (тождественное).}
		\end{enumerate}
	\end{definition}
	\begin{definition}\label{stepen_def}
	Для любой группы $G$ можно определить {\itстепень} элемента $g\in G$ с целым показателем:
	{\center
	\begin{equation*}
	g^k=
	\begin{cases}
		\underbrace{gg\ldots g}_{k},&k>0\\
		e, &k=0\\
		\underbrace{g^{-1}g^{-1}\ldots g^{-1}}_{k}, &k<0.
	\end{cases}
	\end{equation*}}
	\end{definition}
	\begin{preposition}\label{stepen_prep}$\forall g \in G~ \forall k, l \in \Integer~ g^kg^l=g^{k+l}$
	\begin{proof}Рассмотрим различные случаи для $k, l$
	\begin{enumerate}
	\item{$k, l > 0$ - очевидно}
	\item{$k > 0, l < 0, k+l>0$:
	\[
	g^kg^l=\underbrace{gg\ldots g}_{k}\underbrace{g^{-1}g^{-1}\ldots g^{-1}}_{l}=\underbrace{gg\ldots g}_{k+l}=g^{k+l}.\]}
	\end{enumerate}
	Остальные случаи рассматриваются аналогично.
	\end{proof}
	\end{preposition}
	\corollary{\label{stepen_prep_cor}$(g^k)^{-1} = g^{-k}$.}
	\definition{$\langle g\rangle$ - {\it циклическая подгруппа, порожденная элементом $g$} - подгруппа степеней элемента $g\in G$ (является подгруппой из определения~\ref{stepen_def}, утверждения~\ref{stepen_prep} и следствия~\ref{stepen_prep_cor})}
	\definition{\label{order_def}Минимальное $m\in \Natural: g^m = e$ - {\itпорядок} элемента $g$, обозначается $\ord{g}$, если $\nexists m: g^m=e$, то $\ord{g}=\infty$.}

	\begin{preposition}\label{order_prep}
	Если $\ord{g} = n$:
	\begin{enumerate}
	\item{$g^m = e\Leftrightarrow n | m$;}
	\item{$g^k=g^l\Leftrightarrow k \equiv l ~\mod{n}$.}
	\end{enumerate}
	\begin{proof}
	\begin{enumerate}
	\item{$m = qn + r$,\quad $0\leq r < n$ $\Rightarrow\footnote{По определению~\ref{order_def}} g^m=(g^n)^q\cdot g^r=g^r=e\Leftrightarrow r = 0;$}
	\item{$g^k=g^l \is g^{k-l}=e\is n|(k-l)\is k\equiv l \mod{n}$.}
	\end{enumerate}
	\end{proof}
	\end{preposition}
	\begin{corollary}\label{order_prep_cor}
	Если $\ord{g}=n$, то $\left|\langle g\rangle\right| = n$
	\begin{proof}$\langle g\rangle=\{e, g, g^2, \ldots, g^{n-1}\}$, и все элементы различны.
	\end{proof}
	\end{corollary}
	\definition{\label{finite_group_order_def}{\itПорядок конечной группы} $G$ - количество элементов в ней, т.е. $\ord{G} = \left|G\right|$}
	\definition{\label{cyclic_group_def}Группа $G$ называется {\itциклической}, если $\exists g \in G: G=\langle g\rangle$. Всякий такой элемент - {\itпорождающий}.}
	\begin{preposition}\label{ord_prep} $\ord{g} = n \implies \ord{g^k} = \frac{n}{(n, k)}$
	\begin{proof}
	\begin{enumerate}
	\item{$(n, k) = d, n = n_1d, k = k_1d: (n_1, k_1) = 1$;}
	\item{$(g^k)^m=e\is n|km\is n_1|k_1m\is n_1 |m$, откуда $\ord{g^k} = n_1$.}
	\end{enumerate}
	\end{proof}
	\end{preposition}
	\corollary{\label{ord_prep_cor}$g^k\in G=\langle g \rangle$ - порождающий $\is (n, k) = 1.$}
	
	\begin{theorem}\label{cyclic_group_theorem}
	Любая бесконечная циклическая группа изоморфна группе $\Integer$, любая конечная циклическая группа порядка $n$ изоморфна $\Integer_n$
	\begin{proof}
	\begin{enumerate}
	\item{$G=\langle g\rangle, \ord{G} = \infty\implies f:\Integer\rightarrow G, k\mapsto g^k$ - изоморфизм;}
	\item{$G=\langle g \rangle, \ord{G} = n.$ Рассмотрим отображение:
	\begin{equation}
	f:\Integer_n\rightarrow G, [k]\mapsto g^k\quad k\in \Integer
	\end{equation}}
	\begin{equation}\label{eq:2}
	[k] = [l]\is k\equiv l \mod{n}\is g^k=g^l
	\end{equation}
	Из \ref{eq:2} следует, что $f$ корректно определено и биективно, $f(k+l)=f(k)f(l)$ получается из утверждения~\ref{stepen_prep}, откуда $f$ - изоморфизм.
	\end{enumerate}
	\end{proof}
	\end{theorem}
	
	
	
	\begin{theorem}\label{subgroup_cyclic_theorem}
	\begin{enumerate}
	\item{Любая подгруппа циклической группы - циклическая}
	\item{В циклической группе порядка $n$ порядок любой подгруппы делит $n$ и $\forall q: q | n \exists ! H$ - подгруппа порядка $q$}
	\end{enumerate}
	\begin{proof}
	\begin{enumerate}
	\item{
	$G=\langle g\rangle$ - циклическая, $H$ - нетривиальная\footnote{Тривиальная подгруппа, очевидно, циклическая} подгруппа $G$. Если для $m~\in~\Natural~\exists g^{-m}\in H$, то $g^m\in H$. Пусть $m$ - минимальное натуральное число такое, что $g^m\in H$. Докажем, что $H = \langle g^m\rangle$. Пусть $g\in H, k = qm+r, 0\leq r<m$, тогда $g^r=g^k(g^m)^{-q}\in H$, откуда по определению $m$ получается, что $r=0$, откуда $g^k=(g^m)^q$.}
	\item{
	Если $|G|=n$, то предыдущее рассуждение при $k = n (g^k=e\in H)$ показывает, что $n=qm$. При этом \begin{equation}\label{eq:3}H = \{e, g^m, g^{2m}, \ldots, g^{(q-1)m}\}\end{equation} и $H$ - единственная подгруппа порядка $q$ в группе $G$. Обратно, если $q |n, n = qm$, то подмножество $H$, определенное уравнением \eqref{eq:3} - подгруппа порядка $q$.}
	\end{enumerate}
	\end{proof}
	\end{theorem}
	\corollary{В циклической группе простого порядка любая неединичная подгруппа совпадает со всей группой.}
	
	\begin{example}
		$(\mathbb{Z}, +)$ - абелева группа по сложению
		\begin{itemize}
		\item{$0 \in \mathbb{Z}$ - нейтральный элемент, т.к. $\forall a \in \mathbb{Z} a + 0 = 0 + a = a$}
		\item{$\forall a \in \mathbb{Z} ~\exists a^{-1} = -a : a + (-a) = (-a) + a = 0$}
		\end{itemize}
	\end{example}
	\begin{example}
		$(\mathbb{Q}^{\times}, \cdot)$ - абелева группа по умножению, где $\mathbb{Q^\times} = \mathbb{Q}\setminus\{0\}$
		\begin{itemize}
		\item{$1 \in \mathbb{Q^\times}$ - нейтральный элемент, т.к. $\forall a \in \mathbb{Q^\times} ~a \cdot 1 = 1 \cdot a = a$}
		\item{$\forall a \in \mathbb{Q^\times} ~\exists a^{-1} = \frac{1}{a} : a\cdot\frac{1}{a} = \frac{1}{a}\cdot a = 1$}
		\end{itemize}
	\end{example}
 	\example{$GL_n(\mathbb{R})$\footnote{Название произошло от 'General linear group'.} - группа невырожденных\footnote{Для тех, кто не помнит: матрицы с ненулевым определителем.} матриц по умножению\footnote{Из курса алгема: $\forall A: \det{A} \neq 0 \Rightarrow \exists A^{-1}: A A^{-1}=A^{-1}A=E$, где $E$ - единичная, и $\det{AB} = \det{A}\cdot \det{B}$.}.}
 	\example{$SL_n[\Real]$\footnote{Название от 'Special linear group', является подгруппой $GL_n(R)$.} $\subset GL_n[\Real]:= \forall A \in SL_n[\Real]~\det{A} = 1$}
 	\example{$(S_n, \circ)$\footnote{Название от 'Symmetric group'.} - группа перестановок элементов вида $\left\{1,\ldots, n\right\}$, рассматриваемых как функции $\{1,\ldots,n\}\rightarrow S_n$. $\circ$ - операция композиции функций. Является группой, т.к. есть тождественная перестановка и у каждой перестановки есть обратная. Также следует заметить, что $S_n$ подходит под определение~\ref{transformation_group_def}, поэтому можно задать действие $S_n$ на любом конечном множестве.}
 	\example{$D_{2n}$ - группа Диэдра - группа симметрий правильного $n$-угольника $A_1, \ldots, A_n$, включающая поворот и отражение. Состоит из $2n$ элементов:
 	\[
 	\{1, r, r^2, \ldots, r^{n-1}, s, sr, sr^2, \ldots, sr^{n-1}\},
 	\]
 	где $r$ - поворот $n$-угольника на $\frac{2\pi}{n}$, а $s$ - отражение относительно $OA_1$, где $O$ - центр фигуры. Таким образом, $rs$ - повернуть и отразить (читаем слева направо, как композиция функций). В частности, $r^n=s^2=1$ и $r^ks=sr^{-k}$.}
 	\example{$\{1\}$ - тривиальная группа.}
 	\example{В группе $\Integer$ любая подгруппа имеет вид $n\Integer$, где $n > 0$}
 	
 	\newpage
 	\section{Смежные классы по подгруппе, индекс подгруппы. Теорема Лагранжа.}
 	
 	\definition{\label{equiv_def}$G$ - группа, $H \subseteq G$ - подгруппа, тогда $g_1$ и $g_2$ {\it сравнимы по модулю} $H$, если $g_1 \equiv g_2 \mod{G}\is g_1^{-1}g_2\in H \is \exists h \in H: g_2=g_1h$.}
 	\begin{preposition}
 	\label{equiv_prep}
 	Отношение из определения~\ref{equiv_def} - отношение эквивалентности.
 	\begin{proof}
 	\begin{enumerate}
 	\item{$g\equiv g \mod{H}: g^{-1}g=e\in H$;}
 	\item{$g_1\equiv g_2 \mod{H} \implies g_2\equiv g_1 \mod{H}:
 	g_2^{-1}g_1=(g_1^{-1}g_2)^{-1}\in H$;}
 	\item{$g_1\equiv g_2 \mod{H} \land g_2\equiv g_3\mod{H} \implies g_1\equiv g_3 \mod{H}: g_1^{-1}g_2\in H, g_2^{-1}g_3\in H: g_1^{-1}g_3=g_1^{-1}g_2g_2^{-1}g_3=(g_1^{-1}g_2)(g_2^{-1}g_3)\in H$.}
 	\end{enumerate}
 	\end{proof}
 	\end{preposition}
 	
 	\definition{\label{equiv_classes_def}Классы эквивалентности $gH=\{gh | h\in H\}$из определения~\ref{equiv_def} называются {\it левыми смежными классами} по подгруппе $H$, $Hg = \{hg | h \in H\}$ - {\it правыми смежными классами.}}
 	\begin{remark}{$g\mapsto g^{-1}$ - биекция между левыми и правыми смежными классами.}
 	\begin{proof}
 	\[
 	(gH)^{-1}=Hg^{-1}.
 	\]
 	\end{proof}
 	\end{remark}
 	\begin{preposition}
 	\label{equiv_classes_not_intersect_prep}
 	$H\subseteq G, \forall a, b\in G~aH\cap bH \neq \emptyset \implies b\in aH.$
 	\begin{proof}
 	$\forall a, b\in G~aH\cap bH \neq \emptyset \is b^{-1}a\in H\is aH=bH\is b\in aH.$
 	\end{proof}
 	\end{preposition}
 	
 	\definition{Индексом подгруппы $H$ группы $G$ называется число смежных классов $G$ по $H$, обозначается $\left|G : H\right|$.}
 	\begin{theorem}
 	\label{lagrange_theorem}(Лагранжа)
 	$G$ - конечная группа, $H\subseteq G\implies \left|G\right|=\left|G:H\right|\left|H\right|.$
 	\begin{proof}
 	\begin{enumerate}
 	\item{$\forall X=gH\implies\left|X\right|=\left|H\right|$ (очевидно, так как в $H$ все элементы различны, $gH$, полученный умножением всех $h\in H$ на $g$, имеет ту же мощность, если бы это было не так, получилось бы, что мощность $H$ меньше, чем была).}
 	\item{Смежные классы образуют разбиение $G$ на классы эквивалентности, поэтому $\left|G\right|$ есть произведение размерности каждого класса эквивалентности (у всех $|H|$), на их число, т.е. индекс $G$ по $H$.}
 	\end{enumerate}
 	\end{proof}
 	\end{theorem}
 	\corollary{\label{lagrange_theorem_cor_order}
		Порядок любой подгруппы конечной группы делит порядок группы.
 	}
 	\begin{corollary}\label{lagrange_theorem_cor_order_element}
		Порядок любого элемента подгруппы конечной группы делит порядок группы. 	
		\begin{proof}
		Вытекает из следствия~\ref{lagrange_theorem_cor_order} и того, что порядок элемента равен порядку порождаемой им циклической подгруппы.
		\end{proof}
 	\end{corollary}
 	\begin{corollary}\label{lagrange_theorem_cor_simple}
		Всякая конечная группа простого порядка является циклической. 
		\begin{proof}
		Следствие~\ref{lagrange_theorem_cor_order}$\implies$ такая группа совпадает с циклической подгруппой, порожденной любым элементом, не равным $e$.
		\end{proof}			
 	\end{corollary}
 	
 	\section{Гомоморфизмы групп, ядро и образ гомоморфизма. Нормальные подгруппы, факторгруппа. Теоремы о гомоморфизмах.}
 	
 	
\end{document} 

\begin{theorem}
	\begin{enumerate}
	\item{}
	\end{enumerate}
	\begin{proof}
	\begin{enumerate}
	\item{
	
	;}
	\item{
	
	.}
	\end{enumerate}
	\end{proof}
	\end{theorem} 
